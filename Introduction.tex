\section{序論}

本論文では、与えられた命題論理式が充足不能であることの短い証明を作る方法を与える。\cite{Kamo00}

本研究の背景にはSAT solverの高速化という課題がある。
充足可能性問題SATを利用した問題解決手法はシステム検証やプランニング問題など様々な分野において利用されており、
この充足可能性問題を解くSAT solverは広く利用されている\todo{実際に利用した文献}。
しかし充足可能性問題はNP完全であることが証明されており\todo{証明した文献(cook?)}、
問題によってはその解を求めるのに極めて時間がかかることもある。
そのため高速化は重要な課題となっており高速化のための研究が進められている。
近年の高速ソルバの多くはDPLL手続き\todo{DPLL手続き}に加えて矛盾からの節学習を採用したCDCLソルバ\todo{CDCLソルバ}が使用されており、
このソルバに対して改良を行うことで高速化を図る研究などがある。
その1例としてニューラルネットワークを利用してUNSAT coreを予測することで既存ソルバを高速化させる研究がある\todo{UNSAT coreの論文}
また、SAT competitionと呼ばれるSAT solverの性能を競う大会も毎年開催されており、
2024年の大会においてはCDCLソルバの中でも高速なKissat\todo{Kissat}を改良したソルバーなどが多く使用されている。%Main trackにおいてはIsaSATが高い評価を得ている\todo{IsaSATの文献}。

現在でも少しずつだが着実に高速化が進んではいるものの、10倍や100倍といった劇的な改善はなく、
そもそも劇的な改善が可能なのかも分からない。
SAT solverはどこまで高速になりうるか、どうしたら高速化できるか、というのは重要な問いである。

短い証明を作ることは、この問いに関する間接的な答えを与えることができる。
SAT solverは充足不能な命題に対して充足不能性の証明を与えるため、
証明の長さの下限は理想的なSAT solverの実行時間の下限となる。 
もし現在のSAT solverの生成する証明よりも短い証明が存在しないのであれば、
現在のSAT solverはある意味で最適に近いということになる。
逆に1\%の長さの短い証明があれば、現在のSAT solverに対して100倍程度の高速化の可能性があることになる。
また、もし短い証明を構成できれば、それを分析することを通じて高速化の方針を得られる可能性もある。

更にUNSATにおける解となる証明はSATにおける解となる変数選択への割当に比べて、
その作成方法といった部分において多くの方法が存在するため興味深い。
例えば学習節の作り方が変わるとその場合の証明も大きく変わる。
また、SATは変数の数が上限になっているので想定される解の種類が証明よりも少なく、
同じ割当だとしてもその割当にいたる探索の方法が大きく違う場合も十分ありえる。
つまり、探索の方法を変えた際に証明の方がよりその変化の影響を受けやすい。

問題の構成から短い証明を作った鳩の巣問題
問題の構成がわかっている場合にそこから最小の証明を作成する研究などは存在

しかしながら、短い証明を作る研究はこれまでなかった。
SAT solverの研究は多いが重視される指標は速さであり、速く作れる証明が必ずしも短いとは限らないためである。
なお、問題の情報や構成がわかっている場合において短い証明を作成する研究は存在する。
例として鳩の巣原理に対してその最短の証明を作成する研究\todo{鳩の巣問題}がある。

本研究では CDCL solver の中でも比較的改造が簡単な MiniSat\todo{minisat}に対して改造を行い、
特定のタイミングで選択する変数を変更したり、変数の重みを変更したりするなど、変数選択へ介入を行った。
そして、最適化のために遺伝アルゴリズム\todo{遺伝アルゴリズム}を使用し、短い証明が作れるような重み変更のタイミングとその重みの解を探索することで様々な問題に対して短い証明を作成した。
この時の証明の形式は標準的な証明の形式であるDRAT\todo{DRAT}を使用した。

実験の結果として、次のような観察を得た。
遺伝アルゴリズムを変数選択への介入を行うMiniSatに使用することで使用する前のMiniSatよりも短い証明を作れることがわかった。
しかし、ほとんどの問題において現在の高速なSAT solverであるKissatが作る証明よりも短い証明は作れておらず、短い証明ができてもKissatが作る証明の100倍短くなるような大幅に短い証明はできていない。
加えて、特定のタイミングで変数選択を変更しても重みが変わっていなことによって探索が介入をしなかった場合の探索とほとんど同じになってしまうことが得られた。

今回作成した証明がKissatに勝てない理由としては考えられる可能性は3種類ある。
1つ目が今回提案したような変数選択への介入ではKissatに勝てない可能性であり、
この場合についてはMiniSatとKissatの証明の作り方に違いがある可能性がある。
2つ目は遺伝アルゴリズムには効果がない可能性である。
この可能性について調べるためにはパラメータやオペレータを変更した際に効果が変わるかを調べる必要がある。
3つ目はKissatの証明が最適解である可能性であり、
この場合Kissatに対して介入を行なった場合に証明をを短くできるかどうかが課題になってくる。