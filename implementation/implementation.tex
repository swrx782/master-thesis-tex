\section{実装}





実験を行うにあたっていくつかの実装を行なった。

% どういうことをしたかを書いたけどここでは不要なので一旦コメントアウト
\begin{comment}
まず最初にminisatに対しての2種類のオプションを可能にする改造を行なった。
どちらのオプションもファイル名を指定して使用するオプションである。
1つ目のオプションが使用された場合、問題を解く前にminisatは指定されたファイル名のファイルを新規で作成する。
minisatが問題を解く中で新しく節を作成した場合に指定したファイルにその節を追加したという情報を書き込み、
節を削除した場合には指定したファイルその節を削除したという情報を書き込む操作が行われるようにminisatを改造した。
2つ目のオプションが使用された場合、問題を解く前にminisatは指定されたファイルを読み込んでそこからどのような介入を行うかのデータを記憶する。
この介入は何回目の変数選択の際にどのように変数のスコアを書き換えるかという情報を表している。
そして、minisatが問題を解く中で変数選択の回数が指定した回数になった場合、指定した方法で変数のスコアを書き換えるようにminisatを改造した。

次に遺伝アルゴリズムを実行するプログラムの作成を行なった。
このプログラムはCDCLソルバーとUNSATな問題とその他遺伝アルゴリズムに必要なパラメータの情報を受け取る。
これらを受け取った後、プログラムは介入の列で表現される染色体からなる初期の集団を作成し、
そこから交叉と突然変異を用いながら、
指定した問題を指定したCDCLソルバーが解いた際に短い証明を作成できるような介入列の探索を行う。
\end{comment}

これにより
\begin{itemize}
	\item 変数選択における介入とDRATの証明ファイルの作成が可能になったminisat
	\item 変数選択の際に各変数のスコアをランダムに書き換えることとその復元が可能なCDCLソルバーとUNSATな問題を受け取って、
	      その問題の短い証明を作れるような変数選択における重みの書き換えの介入列を探索する遺伝アルゴリズム
\end{itemize}
の2つのアプリケーションを作成した

以下でこの実装を再現するための詳細な説明を行う。





% 変数選択への介入
\input{implementation/intervention.tex}





% 証明を作る
\subsection{証明を作る}



今回の研究においては証明が必要不可欠であり、使用したいソルバーが証明を作成しない場合、そのままの状態では実験に使用することができない。
minisatには証明を出力してファイルに保存する機能が存在しなかったので既存のminisatを改造することで証明を出力できるようにした。

ソルバーが新しく節の追加や削除を行なっている関数についてその際に証明ファイルに節を追加するように変更を行なった。
この節の追加や削除を行なっている関数は2種類に分類することができる。
それは学習節を作る関数と前処理を行う関数の2種類である。
上述したようにこの証明を出力させる方法としてCDCLソルバーが作成する学習節を順にファイルに追加していくことで証明を作ることができるのだが、
minisatにはこの学習節以外にも新しく節を作成している部分が存在するためこのままでは証明として成立してない。
minisatは問題の解を探索する前に、より効率的に探索をするために前処理という操作を行なっている。
この前処理の段階でminisatは問題にあるいくつかの節を組み合わせたりしながら新しい節を作成している。
そのため、この新しく作成した節も証明の中の節として出力する必要がある。
この2種類の方法で作成される節や削除される節を出力していくことで証明ファイルを作成できるように実装を行なった。
なお、証明ファイルポインタは変数proof\_fileに格納されており、節を出力する際には1番目のリテラルから順に出力している。

まず新しく学習節を作る部分についてだが、minisatが学習節を作っているのはsimp/solver.ccの中の関数searchの中にある関数analyzeの部分である。
関数analyzeは矛盾が起きた節と学習節を入れる変数learnt\_clauseとどのレベルまでバックトラックすべきかを表す変数を受け取り、矛盾の原因を探る。
さらにこの関数は同時にその矛盾から学習節を作成しlearnt\_clauseに代入するため、この変数の情報を証明のファイルに出力した。
関数analyzeの後の関数cancelUntilが終了した後にfor文を用いて変数learnt\_clauseの1番目のリテラルの情報から最後のリテラルの情報まで出力を行なった。
\begin{lstlisting}[caption=関数analyzeの変更(core/solver.cc), firstnumber=296]
	// CONFLICT
	// 矛盾の発生
    conflicts++; conflictC++;
    if (decisionLevel() == 0) return l_False;

	// 学習節の作成
    learnt_clause.clear();
    analyze(confl, learnt_clause, backtrack_level);
    cancelUntil(backtrack_level);
    
	// 追記部分
	// 作った学習節をファイルに出力
    if (proof_file) {
        for (int j = 0; j < learnt_clause.size(); j++) {
			fprintf(proof_file, "%s%d ", (sign(learnt_clause[j])? "-" : ""), var(learnt_clause[j])+1); 
        }
        fprintf(proof_file, "0\n");
    }

	// 学習節をデータベースに追加(節の長さが1である場合は次の単位伝搬のためのキューに追加)
	if (learnt_clause.size() == 1){
        uncheckedEnqueue(learnt_clause[0]);
    }else{
        CRef cr = ca.alloc(learnt_clause, true);
        learnts.push(cr);
        attachClause(cr);
        claBumpActivity(ca[cr]);
        uncheckedEnqueue(learnt_clause[0], cr);
    }
\end{lstlisting}

次に前処理における新規の節追加と節削除について説明する。

% 注意点(正しいDRATかどうかの確認はしていないこと)
注意点として、今回おこなった実装は必ず証明として成り立つことを保証していない。
前処理における全ての操作を理解し、要所要所の処理をDRATとしてどのように出力すべきかという確認が、
正しく節が作られているかどうかを保証するために必要である。
しかし、前処理に含まれる操作は複雑なものとなっており、その操作全てについて理解ができていない。
今回はほとんど内容を無視するかたちで、前処理が節を作成・削除していると思われている箇所でその節を証明ファイルに追加節・削除節として出力した。
そのためこれで必要な節が足りているかどうかや不要な節を追加しているかどうかはわからない。
実装をおこなってから現時点までの研究において証明を検証するdrat-trimが検証に失敗したという旨のエラー出力は確認されていない。

% minisatの3つの処理
前述したように前処理での全体の動きは複雑なものになっているが、
節に関連する処理としてここではminisatは節を追加する処理と、節を削除する処理と、節の形を変更する処理をおこなっている。

% 1つ目の処理(節を追加する処理)
1つ目の節を追加する処理についてはcore/solver.ccの中の関数addClause\_()の中で行われている。
この関数は追加したい節を受け取ってその節が可能なら節の縮小を行い、その後節を追加したり、
節の長さ(節に含まれる変数の個数)が1である場合は追加せずに変数への割当を行うよう指示を出す。
ここでは節の縮小を行う前に節を証明のファイルに出力し、さらに節の縮小を行なった直後にも節を証明のファイルに出力した。

\begin{lstlisting}[caption=関数addClause\_の変更(core/solver.cc), firstnumber=154]
    bool Solver::addClause_(vec<Lit>& ps)
    {
        assert(decisionLevel() == 0);
        if (!ok) return false;

        // 追記開始
        // 証明部分
        if (proof_file){
            for (int k=0; k<ps.size(); k++){
                fprintf(proof_file, "%s%d ", (sign(ps[k])? "-" : ""), var(ps[k])+1);
            }
            fprintf(proof_file, "0\n");
        }
        // 追記終了

        // 節の縮小
        // Check if clause is satisfied and remove false/duplicate literals:
        sort(ps);
        Lit p; int i, j;
        for (i = j = 0, p = lit_Undef; i < ps.size(); i++)
            if (value(ps[i]) == l_True || ps[i] == ~p)
                return true;
            else if (value(ps[i]) != l_False && ps[i] != p)
                ps[j++] = p = ps[i];
        ps.shrink(i - j);

        // 追記開始
        // 証明部分
        if (proof_file){
            for (int k=0; k<ps.size(); k++){
                fprintf(proof_file, "%s%d ", (sign(ps[k])? "-" : ""), var(ps[k])+1);
            }
            fprintf(proof_file, "0\n");
        }
        // 追記終了

        // 節の追加(節の長さが1である場合は次の単位伝搬のためのキューに追加)
        if (ps.size() == 0)
            return ok = false;
        else if (ps.size() == 1){
            uncheckedEnqueue(ps[0]);
            return ok = (propagate() == CRef_Undef);
        }else{
            CRef cr = ca.alloc(ps, false);
            clauses.push(cr);
            attachClause(cr);
        }

        return true;
    }
\end{lstlisting}

% 2つ目の節を削除する処理
2つ目の節を削除する処理についてはcore/solver.ccの中の関数removeClauseとcore/solver.ccの中の関数removeSatisfiedで行われている。
前者についてはremoveClauseの処理の最初に削除節を証明のファイルに出力した。
\begin{lstlisting}[caption=関数addClause\_の変更(core/solver.cc), firstnumber=212]
    void Solver::removeClause(CRef cr) {
        // 追記開始
        // 証明部分
        if (proof_file) {
            fprintf(proof_file, "d ");
            for (int k = 0; k < ca[cr].size(); k++) {
                fprintf(proof_file, "%s%d ", (sign(ca[cr][k])? "-" : ""), var(ca[cr][k])+1);
            }
            fprintf(proof_file, "0\n");
        }
        // 追記終了
        Clause& c = ca[cr];
        detachClause(cr);
        // Don't leave pointers to free'd memory!
        if (locked(c)) vardata[var(c[0])].reason = CRef_Undef;
        c.mark(1); 
        ca.free(cr);
    }
\end{lstlisting}
後者の関数は全ての節に対して、現在の変数割当において既に満足している節を削除する際に使用される関数である。
この関数は節中の既に偽なリテラルに関してはそのリテラルを削除することで節の長さを縮小する役割も担っている。
ここでの変更として、節の縮小がなされた場合は縮小後の節を追加しつつ縮小前の節を削除し、
節の縮小がなされていない場合は受け取った節を追加する情報を証明ファイルに出力するように変更を行なった。
関数の実行を行う前に文字列型の変数pre\_clauseとブール型の変数changedを定義し、
各節に対して縮小を行う前にその節を削除する場合に証明ファイルに出力される文字列をpre\_clauseに保存し、changedにはFalseを代入した。
節の縮小が1度でもなされた場合にchangedにTrueを代入し、節の縮小の確認が終わったあとに現在の節を証明ファイルに出力するが、
changedの値がTrueである場合には縮小前の節を削除するようにpre\_clauseも証明ファイルに出力するように変更した。
なお受け取った節が現在の割当において真になっている場合は節を削除するが、関数removeClauseを呼び出しているため、
ここでの節の削除の出力はremoveClauseの中で行う。
\begin{lstlisting}[caption=関数removeSatisfiedの変更(core/solver.cc), firstnumber=601]
    void Solver::removeSatisfied(vec<CRef>& cs)
    {
        int i, j;
        // 追記開始
        string pre_clause;  // 節を縮小する前の節を表す文字列
        bool changed;       // 節の変更(縮小)がされたかどうか
        // 追記終了
        for (i = j = 0; i < cs.size(); i++){
            Clause& c = ca[cs[i]];
            // 追記開始
            // 縮小前の節を表す文字列をpre\_clauseに保存
            if (proof_file) {
                pre_clause = "d ";
                changed = false;
                for (int l = 0; l < ca[cs[i]].size(); l++) {
                    pre_clause += (sign(ca[cs[i]][l])? "-" : "") + to_string(var(ca[cs[i]][l])+1) + " ";
                }
                pre_clause = pre_clause + "0";
            }
            // 追記終了
            if (satisfied(c))
                removeClause(cs[i]);
            else{
                // Trim clause:
                assert(value(c[0]) == l_Undef && value(c[1]) == l_Undef);
                for (int k = 2; k < c.size(); k++)
                    if (value(c[k]) == l_False){
                        c[k--] = c[c.size()-1];
                        c.pop();
                        changed = true; // 追記
                    }
                cs[j++] = cs[i];
                // 追記開始
                // 証明部分
                if (proof_file) {
                    // 現在の節(ca[cs[i]])を追加
                    for (int l = 0; l < ca[cs[i]].size(); l++) {
                        fprintf(proof_file, "%s%d ", (sign(ca[cs[i]][l])? "-" : ""), var(ca[cs[i]][l])+1);
                    }
                    fprintf(proof_file, "0\n");
                    // 節が変わっている場合は変更前の節を削除
                    if (changed) fprintf(proof_file, "%s\n", pre_clause.c_str());
                }
                // 追記終了
            }
        }
        cs.shrink(i - j);
    }
\end{lstlisting}

% 3つ目の節を変更する処理
3つ目の処理を行なっているのはsimp/SimpSolver.ccの中の関数strengthenClauseである。
後者の関数は節と節に含まれるリテラルを受け取ってその節からリテラルを削除する関数になっている。
ここではreturn文によって関数を抜ける前に新しくできた節を証明ファイルに追加するように変更した。
関数内では受け取った節の長さに応じてif文を使って分岐処理を行なっているが、どちらの場合でも変更後の節を証明ファイルに出力している。
\begin{lstlisting}[caption=関数strengthenClauseの変更(simp/SimpSolver.cc), firstnumber=203]
    bool SimpSolver::strengthenClause(CRef cr, Lit l)
    {
        Clause& c = ca[cr];
        assert(decisionLevel() == 0);
        assert(use_simplification);

        // FIX: this is too inefficient but would be nice to have (properly implemented)
        // if (!find(subsumption_queue, &c))
        subsumption_queue.insert(cr);

        if (c.size() == 2){
            removeClause(cr);
            c.strengthen(l);
        }else{
            detachClause(cr, true);
            c.strengthen(l);
            attachClause(cr);
            remove(occurs[var(l)], cr);
            n_occ[l]--;
            updateElimHeap(var(l));
        }

        // 追記開始
        // 証明部分
        if (proof_file) {
            for (int k = 0; k < ca[cr].size(); k++) {
                fprintf(proof_file, "%s%d ", (sign(ca[cr][k])? "-" : ""), var(ca[cr][k])+1);
            }
            fprintf(proof_file, "0\n");
        }
        // 追記終了

        return c.size() == 1 ? enqueue(c[0]) && propagate() == CRef_Undef : true;
    }
\end{lstlisting}

% オプションの追加
上記の処理をオプションとして証明を作成する旨のオプション名を受け付けるようにプログラムの変更を行なった。

% 変数の定義
追加した方法は証明を作る際に変更した方法と同じであり、今回は証明を出力するファイル名を保存しておくために、
オプション名をproofで指定して文字列型オプションの変数proofを既存のオプションと同じ位置に宣言した。
これによりコマンドに"-proof=証明ファイル名"を加えることで証明ファイル名を変数proofに保存することができる。

% 証明ファイルポインタの作成
続いて探索の中でこの証明ファイルを使用するために、SimpSolverクラスの変数Sが持つ変数proof\_fileに証明ファイルの情報を入力する。
この変数proof\_fileにはfopen(proof, "wb")の返り値が代入され、オプションが指定されなかった場合はNULLが代入される。
この変数S.proof\_fileを使用して、問題を受け取った後にソルバーが新規で作成した節や新規で削除した節がある場合は節を証明ファイルに関数printfを使用して出力していく。
削除した節の場合は最初に"d "を出力する。
minisatの節は各要素がリテラルの情報を持つリストになっており、節を出力する際には1番目の要素から順に変数の番号を出力して空白を1文字出力する。
リテラルが変数の否定の形をしている場合は番号の前に"-"をつけて出力する。
この証明ファイルは問題がSATや途中終了であった場合も削除せず残る。
この場合、証明ファイルの中身はプログラムが終了するまでにminisatが出力した補題を表す節の列になっている。
\begin{lstlisting}[caption=オプションを定義するための関数mainの変更(simp/Main.cc), firstnumber=60]
    // Extra options:
    //
    IntOption    verb   ("MAIN", "verb",   "Verbosity level (0=silent, 1=some, 2=more).", 1, IntRange(0, 2));
    BoolOption   pre    ("MAIN", "pre",    "Completely turn on/off any preprocessing.", true);
    BoolOption   solve  ("MAIN", "solve",  "Completely turn on/off solving after preprocessing.", true);
    StringOption dimacs ("MAIN", "dimacs", "If given, stop after preprocessing and write the result to this file.");
    IntOption    cpu_lim("MAIN", "cpu-lim","Limit on CPU time allowed in seconds.\n", 0, IntRange(0, INT32_MAX));
    IntOption    mem_lim("MAIN", "mem-lim","Limit on memory usage in megabytes.\n", 0, IntRange(0, INT32_MAX));
    BoolOption   strictp("MAIN", "strict", "Validate DIMACS header during parsing.", false);

    // 追記開始
    // 証明を吐かせるかどうかのオプション
    StringOption proof          ("MAIN", "proof", "If given, write the proof to this file(proof=column of lemmas).");
    // 追記終了

    parseOptions(argc, argv, true);
        
    SimpSolver  S;
    double      initial_time = cpuTime();

    if (!pre) S.eliminate(true);

    S.verbosity = verb;

    S.proof_file = (proof)? fopen(proof, "wb") : NULL; // 証明ファイルのポインタをメンバ変数に保存 // 追記箇所
        
    solver = &S;
    // Use signal handlers that forcibly quit until the solver will be able to respond to
    // interrupts:
    sigTerm(SIGINT_exit);
\end{lstlisting}
また、新しくメンバ変数proof\_fileを追加するためにクラスの定義部分にメンバ変数を追加している。
この変数は初期値としてNULLが代入されている。
\begin{lstlisting}[caption=メンバ変数の追加(core/Solver.h), firstnumber=120]
    // Extra results: (read-only member variable)
    //
    vec<lbool> model;             // If problem is satisfiable, this vector contains the model (if any).
    LSet       conflict;          // If problem is unsatisfiable (possibly under assumptions),
                                  // this vector represent the final conflict clause expressed in the assumptions.

    // Mode of operation:
    //
    int       verbosity;
    FILE*     proof_file; // 証明ファイルのポインタ // 追加箇所
    double    var_decay;
    double    clause_decay;
\end{lstlisting}
\begin{lstlisting}[caption=メンバ変数の初期化(core/Solver.cc), firstnumber=54]
    Solver::Solver() :

    // Parameters (user settable):
    //
    verbosity        (0)
  , proof_file       (NULL) // 証明ファイルポインタの初期値 // 追加箇所
  , var_decay        (opt_var_decay)
  , clause_decay     (opt_clause_decay)
\end{lstlisting}



\subsubsection{minisatのプログラムの概要}

プログラムは全てC++で書かれており、
\begin{itemize}
	\item ベクトルやランダム関数といった簡単なデータ構造や関数を定義するmtlフォルダ(Mini Template Liblary)
	\item 構文解析やオプション解析といった汎用的な操作を定義するutilsフォルダ
	\item 解の探索といったminisatの核となる操作を定義するcoreフォルダ
	\item より効率的に問題を解くために、問題を解く前に行う前処理を定義するsimpフォルダ
\end{itemize}
の4種類のフォルダから構成される。

実際に問題を解く際にはsimpフォルダの中にあるmain.ccを実行する。
最初にどのようなオプションを受け付けるか(オプション名、オプションの値をどの変数に格納するか、格納する値の型など)を定義し、
関数parseOptionsを呼び出してコマンドからどのオプションが使用されているかを解析し、変数に値を格納する。
次に、ソルバーの状態を表すSimpSolverクラスの変数Sを宣言する。
節の情報といった探索の中で必要な値は全てこのSの中の変数に保存される。
その後、関数gzopenを呼び出して問題のファイルを開き、parse\_DIMACSを呼び出して変数の数や節の数、各節の情報などを各変数に代入する。
そして、関数eliminateを呼び出すことで問題の前処理(簡約化)を行い、その時点でUNSATと判明した場合はUNSATを返し、
そうでない場合は関数solveLimitedを呼び出して問題を解く。
問題を解き終えると、SATである場合は各変数の割当とともにSATを、UNSATである場合はUNSATを出力して終了する。
これがminisatが問題を解く際のおおまかな処理の流れである。

なお関数solveLimitedを呼び出して問題を解く際のminisatのおおまかな動きは以下のようになっている。
最初に単位伝搬によって単位節がある場合に新しい変数に割当を行なっていき、矛盾が発生しなかった(同じ変数に真と偽両方の変数を割り当てなければならない状態にならなかった)場合で、
全ての変数に割当がなされた場合はSATを返す。
まだ未割当の変数がある場合には関数decideによって未割当の変数を1つ選択し、値を割り当てる。
矛盾が発生した場合には関数analyzeによって矛盾が起きた原因を探り学習節を作成する。
もし矛盾がトップレベル(1つも変数選択をしていない状態)で起きたのであればUNSATを返す。
そうでない場合はbacktrack関数によって学習節が単位節になるまで直近の変数への割当を未割当に戻していく操作を行う。
単位節になった時点で止めることで次のループの最初の単位伝搬によって新しく変数に割当をすることができる。





% 遺伝アルゴリズム
\subsection{遺伝アルゴリズム}



ここからは準備で述べた遺伝アルゴリズムを今回の実験においてどのように実装を行うかについて説明する。
プログラムはpython言語を使用した。
以下では証明の長さをMiniSatとdrat-trimを使用した際に出力される証明の長さという意味で使用する。

% 染色体
染色体はタイミングと書き換え方法を表す値を1つの組にしたものを1つの介入として、この介入を並べた介入列を染色体とした。
書き換え方法を表すためにこの値はどのように書き換えたかを表すシード値が使用される。
染色体にはその他の情報として、その染色体が表す介入を行なった際におけ証明の長さを所持している。
この情報は親の選択の際や異なる2つの染色体の優劣の比較に使用される。
もしこの情報が無い場合、証明の長さを取得しようとする度に時間のかかるMiniSatとdrat-trimを実行しなければならず、大幅な時間がかかってしまう。
1度計算を行なったらその情報を所持することによって2回目以降の参照の時間を短縮することができる。
実装においては1つの介入を辞書型の値で表す。
1つ目のキーはタイミングを表す"timing"で、ここにはタイミングを表す整数型の値が入る。
2つ目のキーはシード値を表す"seed"で、ここには再現したい介入のために必要なシード値が入る。
2つ目のシード値についてはMiniSatを実行する前の状態においてはどのような介入かがわからず、シード値は0以上の値であるため、初期値として-1を入れている。
この介入をリストを利用して並べたものを介入列とする。
最後に辞書型を利用して介入列を表す値と証明の長さを表す値を持つ値によって染色体を表現する。
1つ目のキーは介入列を表す"interventions"でその値は各要素が介入列を表す値となるリストであり、
2つ目のキーは証明の長さを表す"proof\_length"でその値は整数型の値となっている。
キー"proof\_length"の値は証明の長さを取得する前は初期値-1が入る。
\begin{flushleft}
             \{ 'interventions'  : [ \{'timing': タイミング,  'seed': シード値\},\\
    \phantom{\{ 'interventions'  : [ \{'timing': タイミング,}\vdots\\
    \phantom{\{ 'interventions'  : [}\{'timing': タイミング,  'seed': シード値\}],\\
    \phantom{\{}'proof\_length'  : 証明の長さ\}
\end{flushleft}
なお介入のタイミングの上限は、介入なしで解かせた時の変数選択数を上限としている。

% 交叉
交叉については、2つの親に対してある1点を指定してそれ以降を交換する1点交叉を使用した。
具体的にはタイミングをランダムに指定して、そのタイミング以降の介入列を交換するようにした。
この処理を行うために1点交叉を行う関数\_crossoverを作成した。
関数\_crossoverは親となる2つの染色体を受け取って、1点交叉により子となる新しい染色体を2つ作成し返す関数である。
引数となる親parentsと返り値となる子childrenはどちらも染色体を要素に持つ長さ2のリストになっている。
子を作成する際には、最初に介入を何も持たない初期化された染色体を2つ作成する。
次に1つ目の親に対して、交叉位置となるタイミング以上での介入は2つ目の染色体の介入リストに加え、
それより前のでの介入は1つ目の染色体に加える。
2つ目の親に対してはその逆の操作を行う。
この状態においては子の染色体における各介入のタイミングが順番になるようになっていない可能性があるので、
sort関数を使用して子の介入の順番を整理する。
交叉位置については関数random.randintを使用してランダムな値を取得する。
下限は2つの親の最初のタイミングの小さい方に1足したものとして設定し、上限は2つの親の最後のタイミングの大きい方で設定した
長さが2以外の染色体のリストが引数として渡された場合はエラーメッセージを出力してプログラムを終了する。
\begin{lstlisting}[caption=1点交叉を行う関数\_crossover]
    # 一点交叉
    def _crossover(parents):

        # 親の数があっているか確認
        if len(parents) != 2:
            print(f"error(_crossover(parents)):", file=stderr)
            print(f" 親の数が不適切です。(2であるはずが{len(parents)}です)",file=stderr)
            exit(1)

        # 交叉位置の下限の設定
        lower_lim_timing = min(parents[0]["interventions"][ 0]["timing"],
                               parents[1]["interventions"][ 0]["timing"]) + 1
        
        # 交叉位置の上限の設定
        upper_lim_timing = max(parents[0]["interventions"][-1]["timing"],
                               parents[1]["interventions"][-1]["timing"])

        # 交叉位置の設定
        crossover_pos = random.randint(lower_lim_timing, upper_lim_timing)

        # 交叉位置の前後で介入情報を交換する
        children = [{'interventions'  : [],
                    'proof_length'    : -1},
                    {'interventions'  : [],
                    'proof_length'    : -1} ]
        for i in range(len(parents[0]["interventions"])):
            j = int(crossover_pos <= parents[0]["interventions"][i]["timing"])
            children[j]["interventions"].append(parents[0]["interventions"][i])
        for i in range(len(parents[1]["interventions"])):
            j = int(parents[1]["interventions"][i]["timing"] < crossover_pos)
            children[j]["interventions"].append(parents[1]["interventions"][i])

        # それぞれの子の介入情報を整理する
        children[0]["interventions"] = list(sorted(children[0]["interventions"],
                                                   key=lambda x: x["timing"]))
        children[1]["interventions"] = list(sorted(children[1]["interventions"],
                                                   key=lambda x: x["timing"]))

        return children, crossover_pos
\end{lstlisting}

% 突然変異
突然変異については、新しい介入を1つ追加する突然変異と今ある介入の中から1つ削除する突然変異の2つを採用した。
突然変異は交叉を行なった後に10\%の確率で行われ、行う場合は追加の突然変異もしくは削除の突然変異のどちらかを等確率で選ぶ。

% 追加の突然変異
追加の突然変異は1つの染色体に対して新しい介入を1つ追加し新しい染色体を作成する。
同一のタイミングでの介入は今回のminisatにおける介入には対応していないため、
この時新しい介入のタイミングは染色体が持っている介入のタイミングと被らないようにしている。
また、今回介入のタイミングは介入なしで解かせた際の変数選択数を上限としているため、
介入数が介入なしの場合の変数選択数と一致して追加できなくなった場合は追加せず終了する。
追加の突然変異は関数\_add\_interventionによって作成した。
この関数は長さ1の染色体のリストを受け取りその染色体の介入列に新しい染色体を追加する。
受け取った染色体自体に介入を追加しているため、もとの染色体は無くなり新しい染色体のみが残る。
介入を追加する際にはあらかじめこれ以上追加できないかどうかを現在の介入数と介入なしの場合の変数選択数をを比較することによって確認を行い、
追加できる場合は現在の介入のタイミングとは被らないタイミングを関数random.randintを使って探す。
そして新しい介入におけるシード値を関数random.randint(0, 4294967295)によって決定し、
これを取得した新しいタイミングと合わせて介入としてリストの最後尾に追加する。
最後に追加した介入が正しい位置になるように介入列をソートする。
介入の回数は変数add\_nによって管理されておりこの変数の回数だけfor文を使って追加の処理を行なっているため、
この変数の値を変更することで追加する介入の回数を変更できる。
現在の各介入におけるタイミングは集合timingsを使って管理する。
\begin{lstlisting}[caption=追加の突然変異を行う関数\_add\_intervention]
    # 突然変異(介入の追加)
    def _add_intervention(parents):

        # 親の数があっているか確認
        if len(parents) != 1:
            print(f"error(_add_intervention(parents)):", file=stderr)
            print(f" 親の数が不適切です。(2であるはずが{len(parents)}です)",file=stderr)
            exit(1)

        # 現在の介入位置の取得
        timings = set([itv["timing"] for itv in parents[0]["interventions"]])
        if len(timings) != len([itv["timing"] for itv in parents[0]["interventions"]]):
            print("error(_add_intervention(parents)):", file=stderr)
            print(" 介入列のタイミングが重複している可能性があります。", file=stderr)
            print(" parents=", file=stderr)
            pprint(parents, stream=stderr)
            exit(1)

        # シード値の設定
        random.seed() # 他でシード値を使う可能性があるので一旦初期化
        seed = random.random()
        random.seed(seed)

        # 指定された数だけ介入を追加する
        add_n = 1
        children = parents
        for _ in range(add_n):

            # これ以上追加できない場合は追加終了
            if len(timings) == decisions: break

            # 新しい介入位置の取得
            new_timing = random.randint(1, decisions)
            while new_timing in timings:
                new_timing = random.randint(1, decisions)

            # 介入の追加
            timings.add(new_timing)
            children[0]["interventions"].append({"seed": random.randint(0, 4294967295),
                                                 "timing": new_timing})
            
        # 染色体の情報の整理
        children[0]["interventions"] = list(sorted(children[0]["interventions"],
                                                   key= lambda x: x["timing"]))
        children[0]["proof_length"    ] = -1

        return seed, add_n
\end{lstlisting}

% 削除の突然変異
削除の突然変異は1つの染色体に対してその中の介入を1つ削除し新しい染色体を作成する。
介入の数が0になった場合は何も行わずに終了する。
削除の突然変異を行う関数\_del\_interventionはほとんどの流れは\_add\_interventionと同じであるが、
異なる点として、削除を行う際には介入のインデックスをランダムに取得し介入が順番になっていることを保ったままそのインデックスにおける介入を削除する。
\begin{lstlisting}[caption=削除の突然変異を行う関数\_del\_intervention]
    # 突然変異(介入の削除)
    def _del_intervention(parents):

        # 親の数があっているか確認(必要なら関数定義しておく)
        if len(parents) != 1:
            print(f"error(_del_intervention(parents)):", file=stderr)
            print(f" 親の数が不適切です。(2であるはずが{len(parents)}です)",file=stderr)
            exit(1)

        # シード値の設定
        random.seed() # 他でシード値を使う可能性があるので一旦初期化
        seed = random.random()
        random.seed(seed)        

        # 指定された数だけ介入を削除する
        del_n = 1
        children = parents
        for _ in range(del_n):

            # これ以上削除できない場合は削除終了
            if len(children[0]["interventions"]) <= 0: break

            # 削除するタイミングのインデックスを取得
            del_index = random.choice(range(len(children[0]["interventions"])))
            children[0]["interventions"].pop(del_index)

        children[0]["proof_length"    ] = -1

        return seed, del_n
\end{lstlisting}

% 評価関数
評価関数については染色体が持つ介入のもとでminisatを実行してからdrat-trimを実行した後にできる証明の長さを返す関数とした。
実装した評価関数evaluateは染色体population\_iを受け取ってminisatとdrat-trimを実行し、
できた証明の長さを受け取った染色体に加えてから受け取った染色体を返す関数である。
染色体の中に1度も行われていない介入がある場合は、その介入のシード値を介入を行なった際に設定したシード値で変更する。
minisatとdrat-trimの実行は関数subprocess.runを使用して実行した。
minisatの実行をする前では染色体の介入の情報を介入ファイルに書き込み、
minisatの実行が終わった後では染色体が持つ介入のシード値を更新する。
ここではminisatが作成した介入のタイミングとそのシード値が書き込まれている受け渡しデータファイル"delivery\_data.txt"を読み取って、
初めて介入を行なった介入のシード値を更新する。
この時すでに介入を行なっている介入に対しては実際に行われた介入のシード値と染色体が持つ介入のシード値が一致しているかどうかチェックする。
実行の中で扱われる4つのファイルは実行前には存在していないことを想定しており、関数の最初にファイルの作成を行い、関数の処理が全て終了した後に全てのファイルを削除する。
また、介入ファイルとdrat-trim前後の証明ファイルはファイル名を変更した場合、minisatやdrat-trimの実行の際に扱うファイル名も一緒に変更されるので処理に問題はないが、
受け渡しデータファイルに関してはminisatが"delivery\_data.txt"という名前で固定してファイルを作成しているため、変更した場合処理に問題が起きる可能性がある。
\begin{lstlisting}[caption=評価関数evaluate]
    # 評価関数
    def evaluate(population_i):

        # ファイルが存在しないかを確認(存在したらエラー)
        def _file_exits_check(file):
            if os.path.isfile(file):
                print(f"error(evaluate(population_i)): 既にファイル{file}が存在します。削除してください。",
                    file=sys.stderr)
                exit(1)

        # 必要な新規ファイルの存在確認と作成
        intervention_file  = WORKING_DIR + "intervention.txt"   # 介入ファイル
        delivery_data_file = WORKING_DIR + "delivery_data.txt"  # 受け渡しデータファイル
        bef_proof_file     = WORKING_DIR + "bef-proof.drat"     # drat-trim前の証明ファイル
        aft_proof_file     = WORKING_DIR + "aft-proof.drat"     # drat-trim後の証明ファイル
        new_files = [intervention_file,
                     delivery_data_file,
                     bef_proof_file,
                     aft_proof_file     ]
        for file in new_files:
            _file_exits_check(file)
            open(file, "w")
    
        # 介入ファイルの作成と介入情報の入力
        with open(intervention_file, "w") as f:
            n_of_interventions = len(population_i['interventions']) # 介入の回数
            f.write(f"i {n_of_interventions}\n")
            for i in range(n_of_interventions):
                interventions_i = population_i['interventions'][i]  # i回目の介入情報
                f.write(f"t {interventions_i['timing']}\n")
                seed = interventions_i["seed"]                      # i回目のシード値
                if seed == -1:
                    f.write( "a Random\n")
                else:
                    f.write(f"a Random_reproduction {seed}\n")
    
        # minisatを実行
        command = [MINISAT, unzip_bench_file,
                   f"-proof={bef_proof_file}", f"-intervention={intervention_file}"]
        execution_result = subprocess.run(command, capture_output=True, text=True)
        if execution_result.returncode != 20:
            print( "error(evaluate(population_i)): ",file=sys.stderr)
            print(f"予期せぬ返り値です: {execution_result.returncode}", file=sys.stderr)
            print(f"問題ファイル名: {unzip_bench_file}", file=sys.stderr)
            print(f"実行結果: ", file=sys.stderr)
            # エラーの情報を得るために実行を全て取得する
            print("arges=")
            pprint(execution_result.args)
            print("returncode=")
            pprint(execution_result.returncode)
            print("stdout=")
            pprint(execution_result.stdout)
            print("stderr=")
            pprint(execution_result.stderr)
            exit(1)
       
        # 各seed値を挿入する(挿入が必要ない場合は確認を行う)
        i = 0
        f = open(delivery_data_file)
        str = f.readline()
        while str:
            data = list(map(int, str.split()))
            if len(data) != 2:
                print(f"error(evaluate(population_i)): {delivery_data_file}の書式が適切ではありません", file=sys.stderr)
                print(f"data: {data}", file=sys.stderr)
                exit(1)
            else: timing, seed = data
            if population_i["interventions"][i]["timing"] != timing:
                print(f"error(evaluate(population_i)): {i+1}回目のタイミングが一致しません", file=sys.stderr)
                print(f"染色体側のタイミング  : {population_i['interventions'][i]['timing']}", file=sys.stderr)
                print(f"実際に介入したタイミング: {timing}", file=sys.stderr)
                exit(1)
            if population_i["interventions"][i]["seed"] == -1:
                population_i["interventions"][i]["seed"] = seed
            else:
                if population_i["interventions"][i]["seed"] != seed:
                    print(f"error(evaluate(population_i)): {i+1}回目のシード値が一致しません", file=sys.stderr)
                    print(f"染色体側のシード値    : {population_i['interventions'][i]['seed']}", file=sys.stderr)
                    print(f"実際に介入した際のシード値: {seed}", file=sys.stderr)
                    exit(1)
            i += 1
            str = f.readline()
        f.close()

        # drat-trimを実行
        command = [MINISAT, unzip_bench_file,
                   f"-proof={bef_proof_file}", f"-intervention={intervention_file}"]
        command = [DRATTRIM, unzip_bench_file, bef_proof_file, "-l", aft_proof_file,]
        execution_result = subprocess.run(command, capture_output=True, text=True)

        # 評価後の証明の長さを更新する
        with open(aft_proof_file) as f:
            population_i["proof_length"] = sum(c[0] != "d" for c in f.readlines())

        # 作成した新規ファイルの削除
        for file in new_files: os.remove(file)

        return population_i
\end{lstlisting}



% 親の取り方
その他細かい部分として新しい子を生成する際の親の選択方法としては、各染色体の評価値の逆数を重みとしてランダムに選択している。
つまり短い証明を作成するような染色体ほど親として選ばれやすいようになっている。
また新しい世代の作り方としては既存の集団の一部分を削除して空いた部分に新しく生成した子を入れて作るのではなく、
既存の集団を全て削除して新しく生成した子で全て入れ替えるようにしている。

% 初期化
また、集団を初期化する際にはランダムに作成している。
決められた介入数に対して介入のタイミングとそのシード値をランダムに設定して各染色体を作成している。
各染色体の介入数については、最初の実装の段階では10で設定しているが、
後述する実験を通じてランダムに選択するかたちに変更した。

% プログラムの枠組み
これらの設定ののもとで遺伝アルゴリズムの実装を行なった。

% パラメータの設定について

% ファイルの圧縮
実装における細かい部分としては、遺伝アルゴリズムの実行後の結果ファイルの作成時に調整を行なった。
遺伝アルゴリズム実行後のデータとして、各世代各染色体の証明の長さをリストにしたデータと各世代各染色体の介入列などの全情報も含めたデータを作成しているが、
後者のデータに関しては染色体の介入列の長さが大きくなった際に全体のデータ量が大きくなってしまう問題がある。
この問題を解消するために各世代各染色体の情報をデータに収めるのではなく、その染色体を作るためにどのような交叉をしたかやどのような突然変異をしたかなどの復元のための情報をデータに収めるようにした。
復元の際に時間がかかるものの、全染色体の介入列の情報など全ての情報を復元によって手に入れることができる。
加えて交叉や突然変異の方法が大きく変わらない(復元のための情報が大きくならない)限り、染色体の介入数がどれだけ大きくなっても一定サイズのデータで出力することができる。